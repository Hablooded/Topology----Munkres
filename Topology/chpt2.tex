\documentclass{article}
\usepackage{fontspec}
\usepackage{ctex}
\usepackage{graphicx}
\usepackage{geometry}
\usepackage{amsmath,amsthm,amsfonts,amssymb}
\usepackage{mathrsfs}
\usepackage{enumitem}
\newtheorem{theorem}{定理}[section]
\newtheorem{proposition}[theorem]{命题}
\newtheorem{lemma}[theorem]{引理}

\newtheorem*{corollary}{推论}
\newtheorem*{theorem*}{定理}
\newtheorem{remark}{注}[section]
\newtheorem{example}{例子}[]
\newtheorem{definition}{定义}[section]


\DeclareMathOperator{\Ker}{Ker}
\DeclareMathOperator{\Ima}{Im}
\DeclareMathOperator{\Aut}{Aut}
\DeclareMathOperator{\ch}{ch}
\DeclareMathOperator{\Gal}{Gal}

\newcommand{\ra}{\rightarrow}
\newcommand{\ds}{\displaystyle}
\newcommand{\mt}{\mapsto}
\newcommand{\xra}{\xrightarrow}

\newcommand{\cT}{\mathcal{T}}


\newcommand{\CC}{\mathbb{C}}
\newcommand{\QQ}{\mathbb{Q}}
\newcommand{\RR}{\mathbb{R}}
\newcommand{\ZZ}{\mathbb{Z}}
\newcommand{\FF}{\mathbb{F}}


\title{\Huge \textbf{Chapter 2}\\
Topology Spaces and Continuous Function}
\date{2025.1.31}
\begin{document}
\maketitle
\newpage

\begin{center}
\section*{拓扑空间和连续函数}
\end{center}

\section{拓扑空间}
\begin{definition}
    集合\(X\)上的拓扑是\(X\)中有下列性质的子集构成的集合族\(\mathcal{T}\):\begin{enumerate}[label=\arabic*.]
        \item \(\varnothing,X\in \cT\).
        \item \(\cT\)的任意子集族的元素的并属于\(\cT\).
        \item \(\cT\)的任意有限子集族的元素的交属于\(\cT\).
    \end{enumerate}
\end{definition}

如果\(X\)是具有拓扑\(\cT\)的拓扑空间,我们说\(X\)的子集\(U\)是\(X\)的开集,若\(U\in \cT\)。根据拓扑的定义,\(\varnothing,X\)是开集,且开集的并和有限交都是开集。则定义了符合条件的开集就定义了拓扑。

\begin{example}
    集合\(X\)的幂集构成的拓扑称为离散拓扑。只包含\(X,\varnothing\)的拓扑称为平凡拓扑。
\end{example}

\begin{example}
    \(X\)是集合。设\(\cT_f=\{U\mid X-U =X\text{,或\(X-U\)是有限的} \}\)。则\(\cT_f\)是\(X\)上的拓扑,称为有限补拓扑。类似地,可以定义\(T_c\),将定义中的“有限”改为“可数”。
\end{example}

\begin{definition}
设 $\mathcal{T}$ 和 $\mathcal{T}'$ 是给定集合 $X$ 上的两个拓扑。如果 $\mathcal{T}' \supseteq \mathcal{T}$,我们说 $\mathcal{T}'$ 比 $\mathcal{T}$ 更细;如果 $\mathcal{T}'$ 真包含 $\mathcal{T}$,我们说 $\mathcal{T}'$ 比 $\mathcal{T}$ 严格更细。在这两种情况下,我们也说 $\mathcal{T}$ 比 $\mathcal{T}'$ 更粗,或严格更粗。如果 $\mathcal{T}' \supseteq \mathcal{T}$ 或 $\mathcal{T} \supseteq \mathcal{T}'$,我们说 $\mathcal{T}$ 和 $\mathcal{T}'$ 是可比较的。
\end{definition}

拓扑的比较不总是必要的。


\section{拓扑的基}

\begin{definition}
    \(X\)是集合。称为在\(X\)上的拓扑的基为\(X\)的子集族\(\mathcal{B}\)(其元素称为基元素)满足:
    \begin{enumerate}[label=\arabic*.]
        \item 对于每个\(x\in X\),至少存在一个基元素\(B\in \mathcal{B}\),使得\(x\in B\).
        \item 设\(B_1,B_2\in\mathcal{B}\),若\(x \in B_1\cap B_2 \),则存在\(B_3\)使得\(x\in B_3\)且\(B_3 \subset B_1\cap B_2\).
    \end{enumerate}
\end{definition}

\begin{example}
    设\(X\)是集合。\(X\)的单点集构成的集族是\(X\)的离散拓扑的基。
\end{example}
\begin{definition}
    
若\(\mathcal{B}\)满足这些条件,我们可以定义\(\mathcal{B}\)生成的拓扑\(\cT\):\(X\)的子集\(U\)称为是在\(X\)上的开集,如果对每一个\(x\in U\),存在基元素\(B\in \mathcal{B}\),使得\(x\in B,B\subset U\)。
\end{definition}
我们将证明这是良定义的。
\begin{proof}
    平凡地,\(\varnothing,X \in \cT\)。考虑\(\cT\)中的指标集族\(\{U_\alpha\}_{\alpha\in J}\),事实上,容易证明\(
    U=\bigcup_{\alpha\in J} U_\alpha \in \cT
    \)。考虑\(U_1,U_2\in \mathcal{T}\),取\(x\in U_1\cap U_2\),则存在\(B_i\),使得\(x\in B_i\subset U_i~(i=1,2)\)。则存在\(B=B_1\cap B_2\),使得\(x\in B,B\subset U_1\cap U_2\)。后递归可得。
\end{proof}



\begin{lemma}
    设\(X\)是集合,\(\mathcal{B}\)是\(X\)上的拓扑\(\cT\)的基,则\(\cT\)是\(\mathcal{B}\)的所有元素的并集的集合族。
\end{lemma}

\begin{lemma}
    设\(X\)是拓扑空间。设\(\mathcal{C}\)是\(X\)中的开集族,满足对于每一个开集\(U\subset X\)和每一个\(x\in U\),存在元素\(C\in \mathcal{C}\),使得\(x\in C\subset U\)。则\(\mathcal{C}\)是\(X\)的拓扑的基。
\end{lemma}

\begin{proof}
    设\(x\in X\),\(X\)是开集,故存在\(C\in \mathcal{C}\),使得\(x\in C\)。
    设\(C_1,C_2\in \mathcal{C}\),若\(x\in C_1\cap C_2\),因为\(C_1,C_2\)都是开集,故\(C_1\cap C_2\)也是开集,则存在\(C_3\in\mathcal{C}\),使得\(x\in C_3\subset C_1\cap C_2\)。
    
    设\(\cT\)是\(X\)的开集族,需说明由\(\mathcal{C}\)生成的拓扑\(\cT'=\cT\)。首先,若\(U\in \cT\),则\(U\)是\(X\)中的开集。则对于每一个\(x\in U\),存在\(C\in \mathcal{C}\),使得\(x\in C\subset U\).故\(U \in \cT'\)。反之,若\(U\in \cT'\),则\(U=\bigcup_\alpha C_\alpha\),其中\(C_\alpha\in \mathcal{C}\),从而是开集。故\(U\in \cT\),得证。
\end{proof}

若需要对拓扑进行比较,可以用拓扑的基来比较拓扑的精细程度。

\begin{lemma}
    设\(\mathcal{B},\mathcal{B}'\)分别为\(X\)上\(\cT,\cT'\)的基,则以下表述是等价的:\begin{enumerate}
        \item \(\cT'\)比\(\cT\)更精细。
        \item 对于\(x\in X\),和每一个包含\(x\)的基元素\(B\in \mathcal{B}\),存在基元素\(B'\in \mathcal{B}'\),使得\(x\in B'\subset B\).
    \end{enumerate}
\end{lemma}

下面我们在实数轴\(\RR\)上定义\(3\)种拓扑。

\begin{definition}
    设\(\mathcal{B}\)是\(\RR\)上所有开区间\((a,b)\)的集合族,其生成的拓扑称为\(\RR\)上的标准拓扑。
    
    设\(\mathcal{B}'\)是\(\RR\)上所有半开区间\([a,b)\)的集合族,其生成的拓扑称为\(\RR\)的下极限拓扑。此时\(\RR\)若具有下极限拓扑,则记为\(\RR_l\)。

    设\(K=\{1/n\mid n\in \ZZ_+\}\),\(\mathcal{B}''\)是所有的开区间\((a,b)\)和所有形如\((a,b)-K\)的集合族,其生成的拓扑称为\(\RR\)上的\(K-\)拓扑。此时\(\RR\)若具有\(K-\)拓扑,记为\(\RR_K\)。
\end{definition}

容易证明这是良定义的。考虑这些拓扑之间的关系。
\begin{lemma}
    \(\RR_l,\RR_K\)的拓扑比\(\RR\)上的标准拓扑更精细,但二者之间不可比较。
\end{lemma}

\begin{definition}
    \(X\)上的拓扑的子基\(\mathcal{S}\)是\(X\)的子集族,其并集为\(X\)。由子基生成的拓扑由\(\mathcal{S}\)的元素的有限交的所有闭集的集合族\(\cT\)定义。
\end{definition}

\section{序拓扑}
\section{积拓扑}
\begin{definition}
    设\(X,Y\)是拓扑空间。\(X\times Y\)上的积拓扑是以\(\mathcal{B}=\{U\times V \mid U,V 
     \text{分别为\(X,Y\)上的开集}\}\)为基的拓扑。
\end{definition}

当\(X,Y\)上的拓扑由基给出时,我们给出积拓扑\(X\times Y\)中更具体的基。
\begin{theorem}
    若\(\mathcal{B,C}\)分别是拓扑空间\(X,Y\)上的基,则集族\[
    \mathcal D=\{B\times C\mid B\in \mathcal{B},C\in \mathcal{C}  \}
    \]是积拓扑\(X,Y\)的基。
\end{theorem}
用副基来表示积拓扑时常是有用的。在这之前,我们先定义投影映射。\begin{definition}
    设\(\pi_1:X\times Y\ra X\),定义为\[
    \pi_1(x,y)=x
    \]类似地,定义\(\pi_2:X\times Y\ra Y\),定义为\[
    \pi_2(x,y)=y
    \]
\end{definition}

注意到,投影映射总是满射。若\(U\subset X\)是空集,则\(\pi_1^{-1}(U)=U\times Y\)也是\(X\times Y\)上的空集。

\begin{theorem}
    集族\[
    \mathcal{S}=\{\pi_1^{-1} (U)\mid U\text{是\(X\)中的开集}\} \cup \{\pi_2^{-1} (V)\mid V\text{是\(Y\)中的开集}\}
    \]
    是积拓扑\(X\times Y\)的副基。
\end{theorem}
\begin{proof}
    注意到,\(U\times V =\pi_1^{-1}(U)\cap \pi^{-1}_2(V)\),这一点可以证明积拓扑含于\(\mathcal{S}\)生成的拓扑。
\end{proof}
\section{子空间}

\begin{definition}
    设\(X\)是带有拓扑\(\cT\)的拓扑空间。若\(Y\)是\(X\)的子集,集族\[
    \cT_Y=\{Y\cap U \mid U\in \cT\}
    \]是\(Y\)上的拓扑,称为子空间拓扑。带有这个拓扑,\(Y\)称为\(X\)的子空间;它的开集由\(Y\)和\(X\)的所有的开集的交构成。
\end{definition}

\begin{lemma}
    若\(\mathcal{B}\)是拓扑空间\(X\)的基,则集族\[
    \mathcal{B}_Y =\{B\cap Y \mid B\in \mathcal{B} \}
    \]是\(Y\)上的子空间拓扑的基。
\end{lemma}
\(X\)和子空间\(Y\)上的开集一般是不等同的,叙述上通常会进行区分。

\section{闭集、极限点}
\section{连续函数}
\section{度量空间}
\end{document}